\chapter{Quick getting started}

\section{Function declarations}
A function is a mapping between an input and output domain. F\# is a functional first programming language, and offers a number of alternative methods for specifying parameters. A simple example is,
\fs{syntaxNotesFunctionDeclarationMul}{}
which declares a function of a tuple and returns their multiplication. The types are inferred from its first use in the second line, i.e., \texttt{mul} is \texttt{val mul : x:int * y:int -> int}.  An argument may be of generic type for input, which need not be inferred without sacrificing type safety, e.g.,
\fs{syntaxNotesFunctionDeclarationGeneric}{}
Here the function \texttt{second} does not use the first element in the tuple, \texttt{x}, and the type of the second element, \texttt{y}, can safely be anything.

Functions may be anonymously declared using the \texttt{fun} keyword,
\fs{syntaxNotesFunctionDeclarationAnonymous}{}
Anonymous functions are often used as arguments to other functions, e.g.,
\fs{syntaxNotesFunctionDeclarationAnonymousAdvanced}{}
This is a powerfull concept, but can make programs hard to read, and overly use is not recommended.

Functions may be declared using pattern matching, which is a flexible method for declaring output depending on conditions on the input value. The most common pattern matching method is by use of the \texttt{match with} syntax,
\fs{syntaxNotesFunctionDeclarationMatchWith}{}

A short-hand only for functions of 1 parameter is the \texttt{function} syntax,
\fs{syntaxNotesFunctionDeclarationFunction}{}
Note that the name given in the match, here \texttt{n}, is not used in the first line, and is arbitrary at the line of pattern matchin, and may even be different on each line. For these reasons is this syntax discouraged.

Functions may be declared from other functions
\fs{syntaxNotesFunctionDeclarationTupleCurrying}{}
For functions of more than 1 argument, there exists a short notation, which is called \idx{currying} in tribute of Haskell Curry,
\fs{syntaxNotesFunctionDeclarationCurrying}{}
Here \texttt{mul 2.0} is a partial specification of the function \texttt{mul x y}, where the first argument is fixed, and hence, \texttt{double} is a function of 1 argument being the second argument of \texttt{mul}. Currying is often used in functional programming, but generally currying should be used carefully, since currying may seriously reduce readability of code.


%%% Local Variables:
%%% TeX-master: "fsharpNotes"
%%% End:
